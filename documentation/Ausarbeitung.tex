% ----------------------------------------------------------------------------
% Projektbericht
% ----------------------------------------------------------------------------
\documentclass[12pt]{report}

% =======================================================
\newcommand{\MyName}{Niklas Deworetzki}
\newcommand{\MyTitle}{CS5341~--~Kernel-Architekturen in Programmiersprachen}
\newcommand{\MyTopic}{The Spineless Tagless G-Machine}        % Thema (Unterüberschrift)
% =======================================================

\usepackage[english, ngerman]{babel} % Automatische Worttrennung
\usepackage[babel]{csquotes}         % Anführungszeichen

\usepackage[T1]{fontenc}             % Umlaute und Sonderzeichen
\usepackage[utf8]{inputenc}          % für Eingabe und Ausgabe
\usepackage{scrhack}                 % unterdrückt Fehlermeldung von listings
\usepackage{parskip}

\usepackage{amsmath}                 % Mathematik
\usepackage{amssymb}                 % Mathematische Symbole
\usepackage{graphicx}                % Grafiken
\graphicspath{{images/}}             % Bilder werden aus images geladen
\usepackage{longtable}               % Mehrseitige Tabellen
\usepackage{wrapfig}                 % Text neben Figuren
\usepackage{multicol}                % Mehrteilige Seiten

% Formatierungen für Grafiken und Elemente.
\usepackage[export]{adjustbox}
\usepackage[a4paper]{geometry}
\usepackage[titles,subfigure]{tocloft}
\usepackage{float}

% Formatierungen für Text, Überschriften, Inhaltsverzeichnis, etc.
\usepackage{fancyhdr}
\usepackage{tocbibind}
\usepackage{blindtext}
\usepackage{helvet}

\usepackage[pdftex,
pdfauthor={\MyName{}},
pdftitle={\MyTitle{}~--~\MyTopic{}}]{hyperref}

% BibLatex configuration
\usepackage[backend=bibtex]{biblatex}
\usepackage[toc,acronyms]{glossaries}
\bibliography{bibliography.bib}     % Pfad zur Datei, die Referenzen enthält

% Verschiedene Grafiken, Figuren, Listings
\usepackage[table]{xcolor} % Definiere eigene Farben
\usepackage{textcomp}      % Sonderzeichen (für Operatoren in Code, z.B.)
\usepackage{subfigure}     % Teilfiguren
\usepackage{filecontents}  % Lade Dateien (für Codebeispiele)
\usepackage{listingsutf8}  % Codeblöcke mit UTF-8 (Umlaute)
\usepackage{lstautogobble} % Entfernt führende Leerzeichen in Codeblöcken
% lstautogobble=false zum deaktivieren für einzelne Codeblöcke

% Grafikbibliothek zum Erstellen eigener Diagramme
\usepackage{tikz}
\usetikzlibrary{decorations.pathreplacing,shapes,arrows,positioning}


%------------------------------------------------------------------------------
% THM Farbdefinitionen
\definecolor{cdmain1}{RGB}{128, 186, 36}
\definecolor{cdmain2}{RGB}{74, 92, 102}
\definecolor{cdhighlight1}{RGB}{156, 19, 46}
\definecolor{cdhighlight2}{RGB}{244, 170, 0}
\definecolor{cdhighlight3}{RGB}{0, 184, 228}
\definecolor{cdhighlight4}{RGB}{0, 40, 120}
%------------------------------------------------------------------------------
\colorlet{fontcolor}{cdmain2}
\colorlet{headercolor}{cdmain1}
%------------------------------------------------------------------------------
% Schrift für Fußnoten
\renewcommand{\footnotesize}{\fontsize{10pt}{12pt}\selectfont}
%------------------------------------------------------------------------------
% Format für Tabellen
\newcommand\HeaderCell[1]{%
  \multicolumn{1}{c}{\cellcolor{cdmain2}\textcolor{white}{#1}}
}
%------------------------------------------------------------------------------
% Farbdefinitionen für Codeblöcke
\colorlet{codecomment}{cdmain1}
\colorlet{codenormal}{cdmain2}
\colorlet{codelblue}{cdhighlight3}
\colorlet{codekeyword}{cdhighlight1}
\definecolor{backcolor}{rgb}{0.95,0.95,0.92}

\lstdefinestyle{codestyle}{
  backgroundcolor=\color{backcolor},
  commentstyle=\color{codecomment},
  keywordstyle=\color{codekeyword},
  numberstyle=\tiny\color{codenormal},
  stringstyle=\color{codelblue},
  basicstyle=\ttfamily\footnotesize,
  breakatwhitespace=false,
  breaklines=true,
  captionpos=b,
  keepspaces=true,
  numbers=left,
  numbersep=5pt,
  showspaces=false,
  showstringspaces=false,
  showtabs=false,
  tabsize=2,
  frame=single
}
\lstset{style=codestyle}
%------------------------------------------------------------------------------
\makeatletter
\def\@makechapterhead#1{%
  \vspace*{50\p@}%
  {\parindent \z@ \raggedright \normalfont
    \interlinepenalty\@M
    \Huge\bfseries  \thechapter\quad #1\par\nobreak
    \vskip 40\p@
}}
\makeatother
\renewcommand*\cftchapnumwidth{2em}
\renewcommand*\cftsecnumwidth{3em}
%------------------------------------------------------------------------------
\makeglossaries{}
\makeatletter
\newcommand \Dotfill {\leavevmode \cleaders \hb@xt@ .8em{\hss .\hss }\hfill \kern \z@}
\makeatother
\renewcommand*\glspostdescription{\Dotfill}

\newacronym{stg}{STG}{Spineless Tagless G-Machine}
\newacronym{jvm}{JVM}{Java Virtual Machine}
\newacronym{cil}{CIL}{Common Language Infrastructure}
\newacronym{ram}{RAM}{Random Access Machine}
\newacronym{whnf}{WHNF}{Weak Head Normal Form}

%%% Local Variables:
%%% mode: latex
%%% TeX-master: "Ausarbeitung"
%%% End:

%------------------------------------------------------------------------------
% Redefine the fancy page style
%\fancypagestyle{fancy}{
\fancyhf{}
\fancyhead[R]{\thepage}
%\fancyfoot{}
\renewcommand{\headrulewidth}{0pt}
%}

\setlength{\headheight}{20mm}

% Redefine the plain page style
\fancypagestyle{plain}{%
  \fancyhf{}%
  \fancyhead[R]{\thepage}%
  \renewcommand{\headrulewidth}{0pt}% Line at the header invisible
  \renewcommand{\footrulewidth}{0pt}% Line at the footer visible
}

% ------------------------------------------------------------------------------


%------------------------------------------------------------------------------
% Zeilenumbrüche in Links
\expandafter\def\expandafter\UrlBreaks\expandafter{\UrlBreaks\do\-\do\\/}
% Zeilenumbrüche bei Unterstrichen (in langen Variablennamen)
\renewcommand\_{\textunderscore\allowbreak}

\newcommand\cn[1]{\textcolor{red}{\textsuperscript{[citation needed]}}}
% ------------------------------------------------------------------------------



% Offizielles Farbschema für Schrift verwenden:
% \color{fontcolor} % Macht Schrift blass


%==============================================================================
\begin{document}
% front matter
\pagestyle{empty}
%\newgeometry{left=40mm,right=15mm,top=30mm,bottom=20mm}
\pagenumbering{Roman}


\begin{titlepage}
  \begin{center}
    \begin{figure}
      % THM Logo
      \includegraphics[width=.9\textwidth]{LOGO_THM_CG_FB06}
    \end{figure}
  \end{center}
  % Title
  \Large
  \begin{center}
    \vspace{1cm}
    \textbf{\MyTitle{}}\linebreak
    \vspace{1cm}
  \end{center}
  \large
  Thema:\\
  \textbf{\MyTopic{}}

  \normalsize
  \vfill
  % \begin{tabular*}{\textwidth}[t]{l,c,l}
  \begin{center}
    \begin{tabular*}{0.75\textwidth}%
      {@{\extracolsep{\fill}}ll}

      {Vorgelegt von:} & {\MyName{}}\\
      {Matrikelnummer} & {5185551}\\
      {} & {}\\
      {} & {}\\
      {Eingereicht bei} & {}\\
      {Hochschulbetreuer/-in:} & {Prof.\ Dr.-Ing.\ Dominikus\ Herzberg}\\
      {} & {}\\
      {} & {}\\
      {Eingereicht am:} & {\today}

    \end{tabular*}
  \end{center}

  \vfill
\end{titlepage}
%==============================================================================

\addtocounter{page}{1}
\pagestyle{fancy}

\tableofcontents

\newcounter{savepage}
\setcounter{savepage}{\number\value{page}}
\newpage

%==============================================================================

% main matter
\pagenumbering{arabic}



\chapter{Einleitung}

Diese Ausarbeitung ist Teil der Dokumentation eines Projektes aus dem Modul \textit{CS5341~--~Kernel-Architekturen in Programmiersprachen}, welches im Wintersemester 2021/2022 stattfand.

Wie der Name es bereits verrät, liegt der Fokus der Veranstaltung auf Programmiersprachen mit einem kleinen Sprachkern, sogenannten \textit{Kernel-Sprachen}.
Diese Sprachen besitzen zumeist die Möglichkeit, sich mit eigenen Mitteln selbst zu erweitern, um so Schicht für Schicht höhere Abstraktionen aufzubauen, ohne diese explizit bei der Implementierung der Sprache zu unterstützen.
Das macht diese Sprachen nicht nur bei der Implementierung interessant, da die Implementierung eines kleinen Sprachkerns nur mit vergleichsweise geringem Aufwand verbunden ist.
Auch für Anwender sind solche Sprachen interessant, da sie zumeist flexibel sind, über Bibliotheken einfach erweitert werden können und die geringe Zahl der Kernfeatures ein schnelles Erlernen fördert.
Bekannte Vertreter für Kernelsprachen sind die verschiedenen Lisp-Dialekte, welche im Sprachkern lediglich Listen, Funktionen und Funktionsanwendungen bieten, während die restliche Funktionalität über Makromechanismen oder reflexive Programmierung entstehen.
Funktionale Sprachen wie Haskell oder Scala wandeln Quellprogramme mit komplexeren Bestandteil in eine kalkülartige Kernsprache um, welche für Analysen im Compiler und die Ausführung herangezogen wird.
Bei den objektorientierten Programmiersprachen ist Smalltalk als wichtiger Vertreter zu nennen.
Hier werden alle Sprachkonstrukte als Objekte dargestellt, welche sich Nachrichten senden können.

Während in den seminaristischen Vorlesungsstunden der Veranstaltung der Fokus zumeist auf den Mechanismen liegt, die in solchen Programmiersprachen verwendet werden, liegt der Fokus dieses Projektes in der Auseinandersetzung mit einem Programmiersprachenkern und besonders auf dem Ausführungsmodell, das diesem zugrunde liegt.
Die sogenannte \textit{STG-Sprache} wird verwendet, um eine Ausführungsumgebung für Haskell bereitzustellen, weswegen die Sprache nicht auf hoher Flexibilität und Erweiterbarkeit sondern vielmehr auf Maschinennähe und Kontrolle der ausgeführten Operationen zur Laufzeit basiert.
Die Besonderheiten dieser Programmiersprache wurden im Rahmen des Projektes untersucht und eine Implementierung der dazugehörigen Maschine in der imperativen Programmiersprache Java erstellt.

Diese Ausarbeitung ist in vier weitere Teile gegliedert, welche die verschiedenen Abschnitte des Projektes widerspiegeln.

\begin{itemize}
\item Kapitel~\ref{chap:grundlagen} befasst sich mit den Grundlagen hinter der STG-Sprache und der dazugehörigen Maschine.

\item Kapitel~\ref{chap:stg} beschreibt die für die Sprache definierten Sprachkonstrukte und deren Bedeutung zur Ausführungszeit.

\item Kapitel~\ref{chap:implementierung} assoziiert die einzelnen Sprachkonstrukte mit der zugehörigen Semantik sowie der Implementierung dieser für eine virtuelle Maschine in Java.

\item Kapitel~\ref{chap:ergebnisse} fasst die Ergebnisse des Projektes zusammen und reflektiert diese.

\end{itemize}


%%% Local Variables:
%%% mode: latex
%%% TeX-master: "../Ausarbeitung"
%%% End:


\chapter{STG-Sprache}

Sprache gibt Beschreibung der STG-Maschine an.
Hier ist auch Unterschied zu anderen Maschinensprachen deutlich.
Keine Instruktionen, stattdessen wird Programmgraph als funktionales Program beschrieben.

Funktionale Programmiersprache dennoch mit Einschränkungen verbunden, die Ausdrucksstärke reduzieren.
Ziel ist nicht angenehmes Programmieren sondern Nähe zur Semantik.

Verschiedene Sprachkonzepte werden vorgestellt und informell mit Bedeutung in der Maschine verknüpft.

Überblick mit Grammatik.
Aussehen ähnelt Haskell, wird auch so dargestellt.
Verwendung von Einrückung und Zeilenumbrüchen zur Abgrenzung.
Interessante Anmerkung: Keine Typisierung im Vergleich zu Haskell.

% TODO: Grafik Grammatik der STG Sprache + Anmerkung dass in Anhang LR(1) Grammatik verwendet wird.


\section{Lambda}

Viele Besonderheiten:
Freie Variablen (benötigt zur Berechnung von Speicherbedarf der Closure)
Update-Flag

Ansonsten wir gewohnt:
Parameter
Rumpf


\section{Let-Bindings}

Let und Letrec.
Erstellt immer Lambdas, direkter Kontakt zu lazy.

Bedeutung ist Speichern auf dem Heap (Heap-Allokation)


\section{Anwendungen}

Funktionen, Konstruktoren, Primitive Operationen.
Eta-Expansion nicht wie in Haskell. % TODO: Warum?

Argumente immer Atome: Variablen oder Primitive Zahlen.
Erfordert anlegen auf Heap via Let für komplexe Parameter.
Einschränkung der Ausdrucksstärke (tatsächlich Ausdruck eingeschränkt), dafür jedoch beschränkte Komplexität in Maschine.
Bei Aufruf müssen Parameter nicht untersucht werden: Alle direkt in atomarer Form.


\section{Fallunterscheidungen}

Auswertung nur hier.
Zwingend nötig, da WHNF den Konstruktor ``kennt'', um Alternative zu bestimmen.
Somit wichtiges Konstrukt.

Unterscheidung in Algebraische Alternativen und Primitive Alternativen.

Immer vorhanden: Defaults.
Passt niemand wird ein Fehler geworfen.

\section{Primitive}

Warum sind Primitive direkt in der Sprachbeschreibung enthalten?
Macht Konzept rund, keine Mehraufwand für Maschinenworte.
Erhöht Performance, da keine Lazyness und direkt Maschinenworte.

Einpacken in Lazyness nur mit minimaler Abstraktionsebene.

% TODO: Beispiel für + und - 


%%% Local Variables:
%%% mode: latex
%%% TeX-master: "../Ausarbeitung"
%%% End:


\chapter{Implementierung}



%%% Local Variables:
%%% mode: latex
%%% TeX-master: "../Ausarbeitung"
%%% End:


\chapter{Ergebnisse}



%%% Local Variables:
%%% mode: latex
%%% TeX-master: "../Ausarbeitung"
%%% End:



%==============================================================================

% end matter
\pagenumbering{Roman}
\addtocounter{savepage}{1}
\setcounter{page}{\value{savepage}}

% list of tables and figures
\listoffigures

% Bibliography
\printbibliography[heading=bibintoc, title=Literaturverzeichnis]{}

% Appendices
\appendix

\end{document}

%%% Local Variables:
%%% mode: latex
%%% TeX-master: t
%%% End:
