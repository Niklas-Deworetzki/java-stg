
\pagestyle{empty}
\begin{quote}
  \vspace*{4cm}

  \begin{center}
    \textbf{\Large Abstract}
  \end{center}

  \vspace{1em}

  Diese Ausarbeitung beschäftigt sich mit den Besonderheiten der Spineless Tagless G-Machine und der dazugehörigen funktionalen Programmiersprache, welche als Kernelsprache einzuordnen ist.
  Anwendungsbereich der Maschine ist die Ausführung von nicht-strikten funktionalen Programmiersprachen wie etwa Haskell auf gewöhnlicher Hardware.
  Es wird untersucht, wie diese Maschine das Schließen der semantischen Lücke zwischen funktionalen Hochsprachen und primitiven Registermaschinen bewerkstelligt, indem eine virtuelle Maschinenimplementierung in Java erstellt wird.
  Die imperative Programmiersprache Java dient dabei als Abstraktion über der Maschine, die ebenfalls ein imperatives Ausführungsmodell aufweist, dabei aber mühselige Aufgaben übernimmt, die vom eigentlichen Fokus ablenken würden.
\end{quote}


%%% Local Variables:
%%% mode: latex
%%% TeX-master: "../Ausarbeitung"
%%% End:
